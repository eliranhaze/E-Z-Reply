\documentclass[12pt]{article}

\usepackage{palatino}
%\usepackage{kpfonts} % [fulloldstylenums]
\usepackage[]{hyperref}
\usepackage[margin=1in]{geometry}
\usepackage[doublespacing]{setspace}
\usepackage{enumitem}
\setlist[description]{leftmargin=\parindent,labelindent=\parindent} % for indented description environment

\usepackage[en-US]{datetime2}
\DTMlangsetup*{showdayofmonth=false}

% my commands for this paper
\usepackage{commands}

\usepackage[
    authordate,
    backend=biber,
    sorting=nyt, % sort by name, year, title
    footmarkoff, % turn off in-line footnote mark (use superscript instead)
    %dashed=false, % in newest version, currently not working here
    useprefix=true, % for prefixed names such as 'van X'
    doi=false,
    isbn=false,
    url=false,
]{biblatex-chicago}
\addbibresource{references.bib}

\DeclareLabeldate{%
  \field{date}
  \literal{forthcoming}
}

% clear irrelevant fields that the current bib prints
\AtEveryBibitem{\clearfield{note}}

\title{The Epistemic and the Zetetic: A Reply to Friedman}
\author{Eliran Haziza}

\begin{document}

\maketitle

\noindent In ``The Epistemic and the Zetetic'',\footnote{\textcite{friedman_epistemic_nodate}.} Jane Friedman presents a provocative puzzle according to which epistemic norms that permit believing on the basis of one's evidence clash with a zetetic (i.e., inquiry related) norm according to which one ought to take the necessary means to settling one's inquiry. The puzzle has significant consequences for normative epistemology, if, as Friedman suggests, the most promising way forward is to reject that there are epistemic permissions to believe in accordance with the evidence. The aim of this paper is to show an easy and independently motivated way out of Friedman's puzzle.

%%%%%%%%%%%%%%%%%%%%%%%%%%%%%%%%%%%%%%%%%%%%%%%%%%%%%%%%%%%%%%%%%%
\section{The puzzle}\label{sec:puzz}

The following evidentialist norms are threatened by Friedman's puzzle:

\begin{description}
    \item[\ep] If one has excellent evidence for \textit{p} at \textit{t}, then one is permitted to \jud{}.\footnote{The \textit{a} subscript stands for `act'. \ep{} is a norm for forming beliefs rather than a norm for being in a state of belief.}
    \item[\eo] If one has excellent evidence for \textit{p} at \textit{t}, then one ought to \jud{}.
\end{description}
%
Friedman argues that \ep{} and \eo{} conflict with the zetetic instrumental principle:

\begin{description}
    \item[ZIP] If one wants to figure out \q{Q}, then one ought to take the necessary means to figuring out \q{Q}.
\end{description}
%
Friedman observes that in any ordinary inquiry one finds oneself surrounded by evidence unrelated to one's inquiry. Consider such a case: Beth has to figure out what she owes at a restaurant (\q{Q}). A lot of evidence is available to Beth at the restaurant most of which irrelevant to her bill inquiry. Perhaps Beth can acquire some beliefs based on that evidence---e.g., that there are five people at the table to the right, or that a couple just sat down---while doing some of the math for what she owes. But at some point she will have to stop being distracted and just focus on her task. At that point ZIP will require Beth to focus on \q{Q}. But even then \ep{} permits Beth to come to believe those propositions supported by her inquiry-irrelevant evidence. This is especially problematic if we assume with Friedman the following:

\newcommand{\opic}{Joint Satisfiability}
\begin{description}
    \item[\opic{}] If one cannot both $\phi$ and $\psi$ at \textit{t} and one is required $\phi$ at \textit{t}, then one is not permitted to $\psi$ at \textit{t}.\footnote{CITE}
\end{description}
%
Beth cannot both focus on \q{Q} and come to believe that, say, there are five people at the table to her right (\textit{p}). If ZIP requires that Beth focus on \q{Q} at \textit{t}, then she is not permitted to judge \textit{p} at \textit{t}. But if Beth has excellent evidence for \textit{p} at \textit{t}, \ep{} permits her to judge \textit{p} at \textit{t}. Beth is both permitted and not permitted to judge \textit{p} at \textit{t}. Crucially, Friedman argues that ZIP itself is an epistemic norms. Its requirements and permissions are thus epistemic ones. It's not that Beth is permitted in one sense to judge \textit{p} but not in another; she is epistemically permitted and epistemically not permitted to judge \textit{p} at \textit{t}. ZIP and \ep{} are downright inconsistent.

The conflict between ZIP and \eo{} arises in much the same way. At some point ZIP requires that Beth focus on \q{Q}, while \eo{} requires that Beth judge \textit{p}, for every \textit{p} excellently supported by her evidence. Since requirements are also permissions, \opic{} leads to the same contradictory conclusion.

Friedman's argument seems to push us to a difficult dilemma: we'll have to reject evidentialist norms such as \ep{} and \eo{} or reject a simple norm that such as ZIP. But if we reject a very weak norm such as \ep{} that merely permits believing in accordance with the evidence, what's left of evidentialist norms? Friedman suggests that \textit{all} such norms must be rejected as they would conflict with ZIP. Genuine epistemic norms, Friedman argues, are norms of inquiry. As I shall argue, however, the evidentialist as an easy way out of this problem. The evidentialist need not give up epistemic requirements or permissions.

%%%%%%%%%%%%%%%%%%%%%%%%%%%%%%%%%%%%%%%%%%%%%%%%%%%%%%%%%%%%%%%%%%
\section{The structure of evidential norms}\label{sec:struct}

To see the solution to the puzzle, let me begin with a somewhat related problem for the evidentialist norm \eo{}. Here it is again:

\begin{description}
    \item[\eo] If one has excellent evidence for \textit{p} at \textit{t}, then one ought to \jud{}.
\end{description}
%
\eo{} is an evidentialist norm of the kind one might find in \textcite{conee_evidentialism_2004}. But there are reasons to think that \eo{} is too strong. For instance, it demands that one believe everything that follows from one's evidence. If my evidence shows that I have thirty-one browser tabs open (\textit{p}), then \eo{} requires that I form the belief that \textit{p}, but also that I form the beliefs that I have less than thirty-two tabs open (\textit{p}\textsubscript{1}), less than thirty-three (\textit{p}\textsubscript{2}), less than thirty-four, (\textit{p}\textsubscript{3}) and so on, since those are supported by my evidence just as well. That may not be a wise way to use my cognitive resources. But the problem is greater than that. Given how \eo{} is formulated, it requires that I judge \textit{p}, \textit{p}\textsubscript{1}, \textit{p}\textsubscript{2}, \textit{p}\textsubscript{3}, and so on, all at the same time. If one cannot do all of those judgings simultaneously, \opic{} entails that \eo{} is inconsistent. A similar problem is that one may have at a given moment excellent evidence for different and not logically related propositions, e.g., that I have thirty-one tabs open (\textit{p}) and that my neighbours are home (\textit{q}). \eo{} requires that one judge \textit{p} at \textit{t} and judge \textit{q} at \textit{t}, which again seems to conflict with \opic{}. How should the evidentialist deal with these problems? One way is to back off from \eo{} to a norm that issues permissions instead of requirements:

\begin{description}
    \item[\ep] If one has excellent evidence for \textit{p} at \textit{t}, then one is permitted to \jud{}.
\end{description}
%
But this isn't quite right. One of the ideas behind an evidentialist norm such as \eo{} is that one should believe what is supported by one's evidence, and not believe things for which one lacks sufficient evidence. \ep{} doesn't give us that. Fortunately, we don't have to look too far. \textcite{feldman_ethics_2000} suggests that, if one has sufficient evidence for some proposition \textit{p}, evidentialist norms require that one believe \textit{p} \textit{given that one takes any doxastic attitude toward p}. \eo{} can be revised as follows:

\begin{description}
    \item[\eoc] If one has excellent evidence for \textit{p} at \textit{t}, then one ought to \jud{}, if one takes any doxastic attitude toward \textit{p}.
\end{description}
%
The idea is this: evidentialist norms don't tell us to take doxastic attitudes, but they do tell us which ones to take \textit{if} we're taking any. \eoc{} avoids the above demandingness problems and preserves the evidentialist spirit in a way that \ep{} fails to do. Norms of this structure are commonplace. In a conversational context, there's a norm of sincerity. Such a norm doesn't require that I utter \textit{p} sincerely; it requires that I utter \textit{p} sincerely if I utter \textit{p} at all.

Before showing that \eoc{} is compatible with ZIP and thus offers an attractive way out of Friedman's dilemma, a few more clarifications are in place. \eoc{} is a norm that issues \textit{conditional requirements}. In a situation where you have excellent evidence for \textit{p}, \eoc{} requires: judge \textit{p} if you take any doxastic attitude toward \textit{p}. There are theoretical question as to how to understand conditional requirements.\footnote{For helpful discussion, see \textcite[ch.~3]{kiesewetter_normativity_2017}.} Some take the `ought' to take wide-scope over a conditional: \textit{Ought}(\textit{take attitude} $\rightarrow$ \textit{judge}). We need not settle the matter here. It suffices for my purposes to assume that a conditional requirement to $\phi$ if I $\psi$ forbids the act of $\psi$-ing without $\phi$. In the case of \eoc{}, when \textit{p} is excellently supported by the evidence, it is acts of taking a doxastic attitude toward \textit{p} but an attitude other than belief, e.g., suspending judgment on \textit{p} or disbelieving that \textit{p}, which violate the norm. Not taking any attitude toward \textit{p} does not violate \eoc{}.

[See Kiesewetter regarding what is and is not plausible to assume.]

I only assume the following about such requirements. If I have a conditional requirement to $\phi$ if I $\psi$ then: (a) not $\psi$-ing does not violate this requirement; (b) I am forbidden from [$\phi$-ing and not $\psi$-ing]; (c) factual detachment is invalid: the fact that I $\psi$ does not entail that I ought to $\phi$; rather, I am required to either not $\psi$ or to $\phi$.

To be clear, \eoc{} does not imply that it is epistemically permissible to not transition from not having to having a doxastic attitude. Rather, its advice is limited to cases in which a doxastic attitude is taken, and it is compatible with the existence of other norms that do have something to say about the permissibility of such transitions.

To see that \eoc{} is compatible with ZIP, consider Beth's restaurant bill inquiry \q{Q}. Suppose that ZIP requires that she focus on \q{Q} at \textit{t}, and at the same time, she has excellent evidence that there are five people sitting at the table to her right (\textit{p}). While \eo{} required that Beth judge \textit{p} at \textit{t}, and thus conflicted with ZIP and \opic{}, \eoc{} does not. The requirement to focus on \q{Q} at \textit{t} is compatible with \eoc{}'s requirement to judge \textit{p} if Beth takes some doxastic attitude toward \textit{p}. Beth will have satisfied ZIP and not violated \eoc{} by focusing on \q{Q} and taking no doxastic attitude toward \textit{p}.

Here are two worries the reader might have. First, suppose that Beth does take a doxastic attitude toward \textit{p}, but not the right one. For instance, she disbelieves \textit{p} instead of believing it as required. Does \eoc{} not require that Beth then revise her belief? If so, \eoc{} does clash with ZIP. But \eoc{} requires no such thing: the conditional requirement that it issues is conditional on Beth's \textit{taking} an attitude, not on having one. So if Beth has the wrong attitude toward \textit{p}, \eoc{} does not issue a requirement to change it. This does not mean, however, that \eoc{} deems her attitude is rational or epistemically justified. She formed it in violation of \eoc{}, after all. But \eoc{} does not issue the requirement to revise the belief. It will do so only once Beth again considers the matter of \textit{p}. The second worry is that if Beth is in the process if taking a doxastic attitude toward \textit{p}, then \eoc{} will issue the unconditional requirement to judge \textit{p}, i.e., that the attitude she takes will be belief. But ZIP will still require that Beth stop those distractions and focus on her inquiry instead. If so, then \eoc{} conflicts with ZIP after all. But this is not so. Once the process of forming a doxastic attitude has begun, either Beth can stop it or she can't. If she can, then \eoc{} does \textit{not} require that Beth judge \textit{p}. Inferring so would be an instance of the invalid pattern of factual detachment. \eoc{} requires that Beth either stop the process or end it with belief. Beth then can satisfy both \eoc{} and ZIP by ending the process. On the other hand, if Beth cannot stop the process, then ZIP does \textit{not} require that she do so. For Friedman's assumption of \opic{} entails an \textit{ought implies can} thesis:

\newcommand{\oc}{OC}
\begin{description}
    \item[\oc{}] If one ought to $\phi$ at \textit{t}, then one can $\phi$ at \textit{t}.\footnote{To see . . . }
\end{description}
%
So if Beth cannot not take a doxastic attitude toward \textit{p} at \textit{t}, then she cannot be required to do so. Either way, then, ZIP and \eoc{} have compatible requirements.

Those who prefer norms of permissions rather than requirements can similarly replace the unconditional \ep{}. Just as \eoc{} is compatible with ZIP, so is a permissive version:

\begin{description}
    \item[\epc] If one has excellent evidence for \textit{p} at \textit{t}, then one is permitted to \judif{}.
\end{description}
%
\epc{} is weaker than \ep{}. While \ep{} permitted to transition from having no doxastic attitude to having some doxastic attitude, given one's evidence, \epc{} says nothing about the permissibility of such transitions. Rather, it only says that judging \textit{p} is permitted given that one makes the transition. In Beth's case, \epc{} doesn't say that Beth is permitted to attend to the available evidence about nearby tables, so it does not clash with a requirement to focus on \q{Q}.

The conditional evidential norms \eoc{} and \epc{} are both compatible with ZIP and independently motivated. Friedman's puzzle, then, has an easy way out. The evidentialist need not give up evidentialist norms altogether even if she accepts Friedman's case against unconditional epistemic norms.

%%%%%%%%%%%%%%%%%%%%%%%%%%%%%%%%%%%%%%%%%%%%%%%%%%%%%%%%%%%%%%%%%%
\section{What cost?}

Replacing \ep{} with \epc{} concedes that there aren't blanket epistemic permissions to believe in accordance with the evidence---there are only conditional permissions. Friedman writes:

\begin{quote}
    Those sorts of blanket permissions are central to normative epistemology as we know it though, and so rejecting them should force a fairly significant rethink of our current understanding of epistemic normativity. If [we reject them], then we will have to say that there may well be cases in which following our excellent evidence and coming to know will have been a mistake — a thoroughly epistemic mistake. It’s hard to know quite how to think about epistemically problematic knowledge or knowledge acquisition on our current understanding of normative epistemology. (p. 29)
\end{quote}
%
To flesh out this worry a bit more, suppose Beth does what is epistemically impermissible for her and comes to know that there are five people at the table to the right (\textit{p}) at \textit{t} instead of focusing on her counting task at \textit{t}. Beth's coming to know that \textit{p} was epistemically impermissible. In the above passage, Friedman calls knowledge obtain in this way ``epistemically problematic knowledge''. Since Friedman does not say what she means by ``epistemically problematic'', and the only normative notions she discusses are epistemic permissions and epistemic requirements, one way to interpret the worry is as follows: If Beth's coming to know \textit{p} was epistemically impermissible, then her resulting state of knowledge that \textit{p} is epistemically impermissible as well. The same could be said for the case of belief: if Beth comes to believe \textit{p} in a way that is epistemically impermissible, then her resulting belief---even if fully supported by the evidence---is epistemically impermissible as well. But these are troubling thoughts. How can a state of knowledge, or a state of evidence-based belief, be \textit{epistemically} impermissible? If these are indeed the consequences of rejecting \ep{}---regardless of whether it is replaced by something like \epc{}---then Friedman is quite right to suggest that we should seriously rethink our current understanding of normative epistemology.

I shall argue, however, that the worry is misplaced. In particular, I shall argue that Beth's resulting knowledge that \textit{p} is permissible, even if her coming to know \textit{p} was not permissible. First, given Friedman's premises---in particular, ZIP and \opic{}---she cannot plausibly claim that Beth's knowledge that \textit{p} is impermissible. Suppose that at \textit{t} Beth knows that \textit{p} as a result of impermissibly coming to know \textit{p}, which she did instead of focusing on \q{Q} as she was required by ZIP. If Beth is not permitted to know \textit{p} at \textit{t}, then, since requirements and permissions are duals, Beth is required not to know \textit{p} at \textit{t}. Either it is possible for Beth not to know \textit{p} at \textit{t}, or it isn't. If it isn't, then by the \textit{ought implies can} (OC in \S\ref{sec:struct}) which is entailed by Friedman's \opic{} assumption, Beth is not required not to know \textit{p} at \textit{t}, and thus is permitted to know \textit{p} at \textit{t}. If, on the other hand, it is possible for her not to know \textit{p} at \textit{t}, then she is required to do that and lose her knowledge in some way. But whatever that entails, it requires not focusing on \q{Q}, which would violate ZIP. Since ZIP requires Beth to \foc{}, and Beth cannot both \foc{} and lose her knowledge that \textit{p} at \textit{t}, Friedman's \opic{} entails that Beth is \textit{not} permitted to lose her knowledge that \textit{p} at \textit{t}. So she is required not to lose it, and thus she is permitted to know \textit{p} at \textit{t}. Either way, then, Beth's resulting knowledge is permissible, even if her coming to know was not. The same applies to belief.

Second, the inference that Beth's knowledge/belief that \textit{p} is not permissible because her coming to know/believe \textit{p} was not permissible seems to be based on the following:

\newcommand{\pp}{Permission Transfer}
\newcommand{\ppk}{Knowledge \pp{}}
\newcommand{\ppb}{Belief \pp{}}
\newcommand{\ppa}{Act-State \pp{}}
\begin{description}
    \item[\ppk] If one knows \textit{p} as a result of coming to know \textit{p}, then: one is permitted to know \textit{p} only if one was permitted to come to know \textit{p}.
    \item[\ppb] If one believes \textit{p} as a result of judging \textit{p}, then: one is permitted to believe \textit{p} only if one was permitted to judge \textit{p}.\footnote{In her \parencite*[p.~689f]{friedman_teleological_2019}, Friedman argues for a thesis equivalent to \ppb{}, writing that ``if some judgment is impermissible then the resulting belief state is as well.''}
\end{description}
%
\ppk{} and \ppb{} are instances of a more general thesis:

\begin{description}
    \item[\ppa] If one is in state S as a result of $\phi$-ing, then: one is permitted to be in S only if one was permitted to $\phi$.
\end{description}
%
But \ppa{} is false. If a norm deems it impermissible to transition from state S\textsubscript{1} to state S\textsubscript{2}, it does not have to deem it impermissible to remain in state S\textsubscript{2} once there. Suppose I am not permitted to exit my house (A), because of a lockdown that forbids being outside. I am thus also not permitted to enter the house across the street (B), since that would require exiting my house. But if I do move from A to B, violating the lockdown, it is not a violation of the lockdown to remain in B. On the contrary, exiting B would be a further violation of the lockdown.

We should distinguish the permissibility of entering a state and the permissibility of remaining in a state. As the above example shows, remaining in a state S could be permissible even if entering S was not. We should expect this to hold whenever a norm forbids changing states only because of some side effect of that change: the change requires being outside when being outside is prohibited, or takes up resources that should be put to other uses. This is what happens in Beth's case. She was not permitted to come to believe/know \textit{p} because doing so distracted her from her inquiry, but she is nevertheless permitted to retain her belief/knowledge that \textit{p}, because not retaining it means further distraction. The permissibility of acquired belief or knowledge states is thus entirely unaffected by rejecting \ep{} and \eo{} in favor of \epc{} and \eoc{}.

\printbibliography

\end{document}
