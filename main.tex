\documentclass[12pt]{article}

\usepackage{palatino}
%\usepackage{kpfonts} % [fulloldstylenums]
\usepackage[]{hyperref}
\usepackage[margin=1in]{geometry}
\usepackage[doublespacing]{setspace}
\usepackage{enumitem}
\setlist[description]{leftmargin=\parindent,labelindent=\parindent} % for indented description environment

\usepackage[en-US]{datetime2}
\DTMlangsetup*{showdayofmonth=false}

% my commands for this paper
\usepackage{commands}

\usepackage[
    authordate,
    backend=biber,
    sorting=nyt, % sort by name, year, title
    footmarkoff, % turn off in-line footnote mark (use superscript instead)
    %dashed=false, % in newest version, currently not working here
    useprefix=true, % for prefixed names such as 'van X'
    doi=false,
    isbn=false,
    url=false,
]{biblatex-chicago}
\addbibresource{references.bib}

\DeclareLabeldate{%
  \field{date}
  \literal{forthcoming}
}

% clear irrelevant fields that the current bib prints
\makeatletter
\AtEveryBibitem{
    \clearfield{note}
    \clearfield{series}
    \clearfield{urldate}
    \clearfield{urlyear}
    \clearfield{urlmonth}
    \global\undef\bbx@lasthash % remove dash for repeating authors in bib
}
\makeatother

\title{The Epistemic and the Zetetic: A Reply to Friedman}
%\author{Eliran Haziza}
\date{}

\begin{document}

\maketitle

\begin{quote}
    \textbf{Abstract}.
    In ``The Epistemic and the Zetetic'', Jane Friedman presents a puzzle according to which epistemic norms that permit believing on the basis of sufficient evidence are incompatible with norms of inquiry. According to Friedman, the most promising solution is to reject such evidentialist norms and to accept instead that the only genuine epistemic norms are norms of inquiry. This paper argues that there is a much more attractive way out of the puzzle, one that reconciles evidentialist norms with Friedman's norms of inquiry.
\end{quote}
%\begin{quote}
%    [Word count: 3011]
%\end{quote}

\noindent In ``The Epistemic and the Zetetic'',\footnote{\textcite{friedman_epistemic_nodate}.} Jane Friedman presents a provocative puzzle: Epistemic norms that permit believing on the basis of one's evidence seem to be incompatible with the zetetic (i.e., inquiry related) norm according to which one ought to take the necessary means to settling one's inquiry. The puzzle has significant consequences for normative epistemology. Friedman argues that the most promising way forward is to reject that evidence is ever sufficient for epistemically permissible belief, and to accept instead that all genuine epistemic norms are zetetic. Once we reject that evidence ever suffices for epistemic permission, we'll have either to reject that evidence justifies belief, or reject that justified belief is epistemically permissible, both quite unattractive options. The aim of this paper is to show a simple and independently motivated way out of Friedman's puzzle---one that makes evidentialist norms compatible with Friedman's zetetic norm. Section 1 presents the puzzle, and section 2 its solution. Section 3 considers an objection that is suggested by Friedman's argument.

%%%%%%%%%%%%%%%%%%%%%%%%%%%%%%%%%%%%%%%%%%%%%%%%%%%%%%%%%%%%%%%%%%
\section{The puzzle}\label{sec:puzz}

The following evidentialist norms are threatened by Friedman's puzzle:

\begin{description}
    \item[\ep] If one has excellent evidence for \textit{p} at \textit{t}, then one is permitted to \jud{}.\footnote{The \textit{a} subscript stands for `act'. \ep{} and \eo{} are norms for forming beliefs rather than norms for being in a state of belief.}
    \item[\eo] If one has excellent evidence for \textit{p} at \textit{t}, then one ought to \jud{}.
\end{description}
%
Friedman argues that \ep{} and \eo{} conflict with the zetetic instrumental principle:

\begin{description}
    \item[ZIP] If one wants to figure out \q{Q}, then one ought to take the necessary means to figuring out \q{Q}.
\end{description}
%
Friedman observes that in any ordinary inquiry one finds oneself surrounded by evidence unrelated to one's inquiry. To properly inquire, one will have to ignore at least some of that evidence. If so, Friedman argues, there are going to be cases in which \ep{} permits judging \textit{p} on the basis of some evidence, while ZIP requires ignoring that evidence and thus not judging \textit{p}.

To see this in more detail, consider one of Friedman's cases. Suppose Beth is at a restaurant when the bill arrives and she has to figure out what she owes (\q{Q}). \ep{} permits Beth to come to believe all kinds of things on the basis of the evidence she has at the restaurant: that there there are five people at the table to her right, that somebody has ordered wine, that there are more than four windows, and so on. Suppose that Beth does so and does not make progress on her calculation task. Since her inquiry is temporally urgent---she has to figure out \q{Q} soon---there comes a point \textit{t} at which ZIP requires Beth to just focus on \q{Q} and not be distracted anymore. \ep{}, however, still permits Beth to come to believe unrelated things at \textit{t}. The conflict becomes evident if we assume with Friedman the following:

\newcommand{\opic}{Joint Satisfiability}
\begin{description}
    \item[\opic{}] If one cannot both $\phi$ and $\psi$ at \textit{t} and one is required to $\phi$ at \textit{t}, then one is not permitted to $\psi$ at \textit{t}.\footnote{Friedman makes this assumption explicit in p. 14f. I follow Friedman in using `ought' and `required' interchangeably.}
\end{description}
%
Given her ordinary cognitive limitations, Beth cannot focus on her task of figuring out \q{Q} and at the same time judge that there are five people at the table to her right (\textit{p}). If Beth is required by ZIP to focus on \q{Q} at \textit{t}, then, given \opic{}, she is not permitted to judge \textit{p} at \textit{t}. But since Beth has excellent evidence for \textit{p} at \textit{t}, \ep{} permits her to judge \textit{p} at \textit{t}. Assuming, as Friedman does, that norms of inquiry such as ZIP are epistemic norms, and that their requirements and permissions are epistemic ones, we get the result that Beth is both epistemically permitted and epistemically not permitted to judge \textit{p} at \textit{t}. ZIP and \ep{}, and by extension ZIP and \eo{}, are therefore inconsistent.\footnote{Since I accept Friedman's claim that ZIP is epistemic for the purposes of this paper, in what follows I use `permission' and `requirement' to mean `epistemic permission' and `epistemic requirement'.}

Friedman's argument seems to present a difficult dilemma. If we reject \ep{}, we'll have to give up on at least one of the following plausible claims: (a) evidence justifies belief; (b) a justified belief is epistemically permissible. Moreover, since \ep{} is a fairly weak norm, issuing only permissions to believe, we might have to reject all such purely evidential norms. Rejecting ZIP does not seem plausible either. Friedman's favored solution is to reject all evidentialist norms and to take only zetetic norms to be genuinely epistemic. As I shall argue, however, there is a simpler and more attractive solution to the problem---one which gives up neither zetetic nor evidentialist norms.

%%%%%%%%%%%%%%%%%%%%%%%%%%%%%%%%%%%%%%%%%%%%%%%%%%%%%%%%%%%%%%%%%%
\section{The structure of evidential norms}\label{sec:struct}

To see the solution to the puzzle, let me begin with a related problem for the evidentialist norm \eo{}. Here it is again:

\begin{description}
    \item[\eo] If one has excellent evidence for \textit{p} at \textit{t}, then one ought to \jud{}.
\end{description}
%
\eo{} is an evidentialist norm of the kind defended, e.g., in \textcite{feldman_evidentialism_1985}. It's a norm that tells us what we ought to believe---that which is supported by our evidence. But there are reasons to think that \eo{} is too strong. For instance, it demands that one believe everything that follows from one's evidence. If my evidence shows that I have thirty-one browser tabs open (\textit{p}), then \eo{} requires not only that I judge that, but also that I judge that I have less than thirty-two tabs open (\textit{p}\textsubscript{1}), less than thirty-three (\textit{p}\textsubscript{2}), less than thirty-four (\textit{p}\textsubscript{3}), and so on, since those are supported by my evidence just as well. That may not be a wise way to use my cognitive resources. But the problem is greater than that. Given how \eo{} is formulated, it requires that I judge \textit{p}, \textit{p}\textsubscript{1}, \textit{p}\textsubscript{2}, \textit{p}\textsubscript{3}, and so on, all at the same time. If one cannot do all of those infinite judgings simultaneously, \opic{} entails that \eo{} is inconsistent. A similar problem is that one may have at a given moment excellent evidence for several unrelated propositions. If at time \textit{t} I have evidence that there are thirty-one tabs open (\textit{p}) and that my neighbours are home (\textit{q}), \eo{} requires that I judge \textit{p} at \textit{t} and judge \textit{q} at \textit{t}, which again seems to conflict with \opic{}. How should the evidentialist deal with these problems? One way is to retreat to a norm that issues permissions instead of requirements:

\begin{description}
    \item[\ep] If one has excellent evidence for \textit{p} at \textit{t}, then one is permitted to \jud{}.
\end{description}
%
But this isn't quite right. One idea behind evidentialist norms such as \eo{} is that one should not believe things for which one lacks sufficient evidence, and \ep{} doesn't give us that. Fortunately, there is another option. To avoid demandingness problems of the above sort, Richard \textcite{feldman_ethics_2000} suggests that if one has sufficient evidence for some proposition \textit{p}, then one is required to believe \textit{p} \textit{given that one takes any doxastic attitude toward p}. I propose, then, the following revision of \eo{}:

\begin{description}
    \item[\eoc] If one has excellent evidence for \textit{p} at \textit{t}, then: one ought to \jud{}, if one takes any doxastic attitude toward \textit{p}.
\end{description}
%
The idea is this: Evidentialist norms don't tell us to take doxastic attitudes, but they do tell us which ones to take \textit{if} we're taking any. \eoc{} avoids the above demandingness problem. It does not require that I come to believe all of \textit{p}\textsubscript{1}, \textit{p}\textsubscript{2}, \textit{p}\textsubscript{3}, and so on, even though all are sufficiently supported by my evidence. But if I do take some doxastic attitude toward any of them, \eoc{} requires that the attitude I take will be belief and not, say, suspended judgment or disbelief.

\eoc{} is also compatible with ZIP. It thus offers an attractive way out of Friedman's dilemma. Before showing that, however, a few clarifications. \eoc{} is a norm that issues \textit{conditional requirements}. These are requirements of the form: $\phi$ if you $\psi$. In our case: judge \textit{p} if you take any doxastic attitude toward \textit{p}. There are theoretical questions as to how to understand conditional requirements.\footnote{For helpful discussion, see \textcite[ch.~3]{kiesewetter_normativity_2017}.} Some, for instance, take the `ought' to take wide scope over a conditional: \textit{Ought}(\textit{take attitude} $\rightarrow$ \textit{judge}). We need not settle the matter here. It suffices for our purposes to assume that a conditional requirement to $\phi$ if I $\psi$ forbids the act of $\psi$-ing without $\phi$-ing, and does not require $\psi$-ing, both of which I take to be uncontroversial. One thus violates \eoc{} only by taking a doxastic attitude other than belief toward \textit{p}, when one has excellent evidence for \textit{p}. Not taking any doxastic attitude toward \textit{p} does not violate \eoc{}.

To be clear, \eoc{} does not imply that it is epistemically permissible not to take a doxastic attitude toward some \textit{p}. Rather, its advice is limited to cases in which a doxastic attitude is taken, and it is compatible with the existence of other norms that do have something to say about the permissibility of such acts.

To see that \eoc{} is compatible with ZIP, consider Beth's restaurant bill inquiry \q{Q}. Suppose that ZIP requires that she focus on \q{Q} at \textit{t}, and at the same time, she has excellent evidence that there are five people sitting at the table to her right (\textit{p}). While \eo{} required that Beth judge \textit{p} at \textit{t}, and thus conflicted with ZIP and \opic{}, \eoc{} does not. The requirement to focus on \q{Q} at \textit{t} is compatible with \eoc{}'s requirement to judge \textit{p} if Beth takes some doxastic attitude toward \textit{p}. Beth will have satisfied ZIP and not violated \eoc{} by focusing on \q{Q} and not taking a doxastic attitude toward \textit{p}.

At this point, the reader might have two worries. First, one might worry that if the antecedent of the conditional requirement holds, then the consequent will be required. In that case, \eoc{} will issue an unconditional requirement to judge \textit{p}, and thus conflict with ZIP's requirement to focus on \q{Q}. But we should not infer that one is required to $\phi$ from the requirement to $\phi$ if one $\psi$'s and the fact that one $\psi$'s. This pattern of inference, known as \textit{factual detachment}, is generally agreed to be invalid.\footnote{See discussion in \textcite[\S 3.1]{kiesewetter_normativity_2017}.} The reason is simple: If one is required to $\phi$ if one $\psi$'s, and one also $\psi$'s, then one can avoid violating the conditional requirement by not $\psi$-ing.


The second worry is this. Suppose that Beth does take a doxastic attitude toward \textit{p}, but not the right one. For instance, she disbelieves \textit{p} when the evidence supports believing \textit{p}. Does \eoc{} not require that Beth then revise her belief? If so, \eoc{} does clash with ZIP, because revising the belief that \textit{p} is another way of not focusing on \q{Q}. But \eoc{} requires no such thing: The conditional requirement that it issues is conditional on Beth's \textit{taking} an attitude, not on having one. So if Beth has the wrong attitude toward \textit{p}, \eoc{} does not issue a requirement to revise it. (This does not mean, of course, that \eoc{} deems her attitude epistemically permissible or justified.) So even in such cases \eoc{} does not issue requirements that are incompatible with the requirements of ZIP.

Those who prefer norms of permissions rather than requirements can similarly replace the unconditional \ep{}. Just as \eoc{} is compatible with ZIP, so is a permissive version:

\begin{description}
    \item[\epc] If one has excellent evidence for \textit{p} at \textit{t}, then: one is permitted to \judif{}.
\end{description}
%
\epc{} is weaker than \ep{} and issues conditional permissions. While \ep{} permitted to transition from having no doxastic attitude to having some doxastic attitude, given one's evidence, \epc{} says nothing about the permissibility of such transitions. Rather, it only says that judging \textit{p} is permitted given that one makes the transition. In Beth's case, \epc{} doesn't say that Beth is permitted to take doxastic attitudes toward, e.g., the number of people at nearby tables, so it does not clash with a requirement to focus on \q{Q}.

The conditional evidential norms \eoc{} and \epc{} are both compatible with ZIP and independently motivated. We thus have an easy way out of Friedman's puzzle.

%%%%%%%%%%%%%%%%%%%%%%%%%%%%%%%%%%%%%%%%%%%%%%%%%%%%%%%%%%%%%%%%%%
\section{What cost?}\label{sec:cost}

Replacing \ep{} with \epc{} concedes that there aren't blanket epistemic permissions to believe in accordance with the evidence---there are only conditional permissions. This might seem problematic. Friedman writes:

\begin{quote}
    Those sorts of blanket permissions are central to normative epistemology as we know it though, and so rejecting them should force a fairly significant rethink of our current understanding of epistemic normativity. If [we reject them], then we will have to say that there may well be cases in which following our excellent evidence and coming to know will have been a mistake — a thoroughly epistemic mistake. It’s hard to know quite how to think about epistemically problematic knowledge or knowledge acquisition on our current understanding of normative epistemology. (p. 29)
\end{quote}
%
To flesh out this worry a bit more, suppose Beth does what is epistemically impermissible for her and comes to know that there are five people at the table to the right (\textit{p}) at \textit{t} instead of focusing on her calculation task at \textit{t}. Beth's coming to know that \textit{p} was epistemically impermissible. In the above passage, Friedman calls knowledge obtain in this way ``epistemically problematic knowledge''. Since Friedman does not say what she means by ``epistemically problematic'', and the only normative notions she discusses are epistemic permissions and epistemic requirements, one way to interpret the worry is as follows: If Beth's coming to know \textit{p} was epistemically impermissible, her resulting state of knowledge that \textit{p} is epistemically impermissible as well. The same could be said for the case of belief: If Beth comes to believe \textit{p} in a way that is epistemically impermissible, then her resulting belief---even if fully supported by the evidence---is epistemically impermissible as well. These are troubling results. How can a state of knowledge, or a state of evidence-based belief, be \textit{epistemically} impermissible? If these are indeed the consequences of rejecting \ep{}---regardless of whether it is replaced by something like \epc{}---then it has significant costs, and Friedman is quite right to suggest that we should rethink our current understanding of normative epistemology.

But the worry is overstated. In particular, I shall argue that Beth's resulting knowledge that \textit{p} is permissible, even if her coming to know \textit{p} was not. First, given Friedman's premises---in particular, ZIP and \opic{}---she cannot plausibly claim that Beth's knowledge that \textit{p} is impermissible. Suppose that at \textit{t} Beth knows \textit{p} as a result of impermissibly coming to know \textit{p}, which she did instead of focusing on \q{Q}. If Beth is not permitted to know \textit{p} at \textit{t}, then, since requirements and permissions are duals, Beth is required not to know \textit{p} at \textit{t}. Either it is possible for Beth not to know \textit{p} at \textit{t}, or it isn't. If it isn't, then by \opic{} Beth is not required not to know \textit{p} at \textit{t}, and thus is permitted to know \textit{p} at \textit{t}, for \opic{} entails an \textit{ought implies can} thesis:

\newcommand{\oc}{Ought Implies Can}
\begin{description}
    \item[\oc{}] If one ought to $\phi$ at \textit{t}, then one can $\phi$ at \textit{t}.\footnote{To see that \opic{} entails \oc{}, consider the case where $\phi$ = $\psi$. If one cannot $\phi$, and one is required to $\phi$, \opic{} entails that one is not permitted to $\phi$---contradiction. So if one cannot $\phi$, one is not required to $\phi$.}
\end{description}
%
If, on the other hand, it is possible for Beth not to know \textit{p} at \textit{t}, then she is required to do that and lose her knowledge in some way. But that would be a further violation of ZIP. Since ZIP requires Beth to \foc{}, and Beth cannot both \foc{} and do whatever she needs to do in order to lose her knowledge that \textit{p} at \textit{t}, \opic{} entails that Beth is \textit{not} permitted to lose her knowledge that \textit{p} at \textit{t}. So she is required not to lose it, and thus she is permitted to know \textit{p} at \textit{t}. Either way, then, Beth's resulting knowledge is permissible, even if her coming to know was not. The same applies to evidence-based belief.

Second, the inference that Beth's knowledge/belief that \textit{p} is not permissible because her coming to know/believe \textit{p} was not permissible seems to be based on the following:

\newcommand{\pp}{Permission Transfer}
\newcommand{\ppk}{Knowledge \pp{}}
\newcommand{\ppb}{Belief \pp{}}
\newcommand{\ppa}{Act-State \pp{}}
\begin{description}
    \item[\ppk] If one knows \textit{p} as a result of coming to know \textit{p}, then: one is permitted to know \textit{p} only if one was permitted to come to know \textit{p}.
    \item[\ppb] If one believes \textit{p} as a result of judging \textit{p}, then: one is permitted to believe \textit{p} only if one was permitted to judge \textit{p}.\footnote{In her \parencite*[p.~689f]{friedman_teleological_2019}, Friedman argues for a thesis equivalent to \ppb{}, writing that ``if some judgment is impermissible then the resulting belief state is as well.''}
\end{description}
%
\ppk{} and \ppb{} are instances of a more general thesis:

\begin{description}
    \item[\ppa] If one is in state S as a result of $\phi$-ing, then: one is permitted to be in S only if one was permitted to $\phi$.
\end{description}
%
But \ppa{} is false. If a norm deems it impermissible to transition from state S\textsubscript{1} to state S\textsubscript{2}, it does not have to deem it impermissible to remain in state S\textsubscript{2} once there. Suppose I am not permitted to exit my house (A), because of a lockdown that forbids being outside. I am thus also not permitted to enter the house across the street (B), since that would require exiting my house. But if I do move from A to B, violating the lockdown, it is not a violation of the lockdown to remain in B. On the contrary, exiting B would be a further violation of the lockdown.

We should distinguish the permissibility of entering a state and the permissibility of remaining in a state. As the above example shows, remaining in a state S could be permissible even if entering S was not. We should expect this to hold whenever a norm forbids a change of state only because of some feature of the change itself, regardless of the states involved, e.g., a change of houses requires being outside when being outside is prohibited, and a change in a belief state takes up resources that should be put to other uses. This is what happens in Beth's case. She was not permitted to come to believe/know \textit{p} because doing so distracted her from her inquiry, but she is nevertheless permitted to retain her belief/knowledge that \textit{p}, because not retaining it means further distraction. The permissibility of acquired belief or knowledge states is thus entirely unaffected by ZIP or by rejecting \ep{} and \eo{} in favor of \epc{} and \eoc{}.

\printbibliography

\end{document}
