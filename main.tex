\documentclass[12pt]{article}

\usepackage{palatino}
%\usepackage{kpfonts} % [fulloldstylenums]
\usepackage[]{hyperref}
\usepackage[margin=1in]{geometry}
\usepackage[doublespacing]{setspace}
\usepackage{enumitem}
\setlist[description]{leftmargin=\parindent,labelindent=\parindent} % for indented description environment

\usepackage[en-US]{datetime2}
\DTMlangsetup*{showdayofmonth=false}

% my commands for this paper
\usepackage{commands}

\usepackage[
    authordate,
    backend=biber,
    sorting=nyt, % sort by name, year, title
    footmarkoff, % turn off in-line footnote mark (use superscript instead)
    %dashed=false, % in newest version, currently not working here
    useprefix=true, % for prefixed names such as 'van X'
    doi=false,
    isbn=false,
    url=false,
]{biblatex-chicago}
\addbibresource{references.bib}

\DeclareLabeldate{%
  \field{date}
  \literal{forthcoming}
}

% clear irrelevant fields that the current bib prints
\AtEveryBibitem{\clearfield{note}}

\title{The Epistemic and the Zetetic: A Reply to Friedman}
\author{Eliran Haziza}

\begin{document}

\maketitle

\section{Introduction}

In ``The Epistemic and the Zetetic'',\footnote{\textcite{friedman_epistemic_nodate}.} Jane Friedman presents a provocative puzzle according to which epistemic norms that permit believing on the basis of one's evidence clash with a zetetic (i.e., inquiry related) norm according to which one ought to take the necessary means to settling one's inquiry. The puzzle has radical consequences for normative epistemology, if, as Friedman suggests, the most promising way forward is to reject that there are epistemic permissions to believe in accordance with the evidence.

This paper critically examines Friedman's puzzle and argues for two claims. First, contrary to Friedman's suggestion, rejecting epistemic permissions to believe in accordance with the evidence does not resolve the inconsistency that is at the heart of the puzzle. This suggests that the source of inconsistency lies not in a tension between the epistemic and the zetetic, but elsewhere. I argue that the problem arises from a conflict between two of Friedman's assumptions: an \textit{ought implies can} thesis, and the claim that there is a point during inquiry at which one ought to focus on one's inquiry. Second, even if Friedman's argument for the epistemic-zetetic conflict is granted, there are nearby epistemic norms that are both compatible with Friedman's zetetic norm and independently more plausible than the epistemic norms that are purportedly inconsistent with it.

\section{Friedman's argument}\label{sec:2}

An epistemic norm that Friedman suggests might have to be rejected is:

\begin{description}
    \item[\ep] If one has excellent evidence for \textit{p} at \textit{t}, then one is permitted to \jud{}.\footnote{The \textit{a} subscript stands for `act'. \ep{} is a norm for forming beliefs rather than a norm for being in a state of belief.}
\end{description}
%
Friedman argues that \ep{} conflicts with the zetetic instrumental principle:
%
\begin{description}
    \item[ZIP] If one wants to figure out \q{Q}, then one ought to take the necessary means to figuring out \q{Q}.
\end{description}
%
\ep{} seems highly plausible and, Friedman claims, is central to current epistemic theorizing. ZIP is an instance of a more general norm of instrumental rationality. But, if Friedman is right, \ep{} and ZIP cannot both be correct. One can be in a situation where ZIP requires that one focus on one’s inquiry, and, at the same time, one has a lot of unrelated evidence for propositions that \ep{} permits one to come to believe. But coming to believe those things will detract from focusing on one’s inquiry. ZIP and \ep{}, then, appear to pull in different directions. But ZIP is itself an epistemic norm, Friedman argues, and its requirements are epistemic requirements. If so, \ep{} and ZIP don’t just pull in different directions---they are downright inconsistent.

Let us look more carefully at Friedman’s argument. Friedman begins with the following assumption about permissions and requirements:\footnote{I assume with Friedman that ZIP issues \textit{epistemic} requirements, so in what follows, all talk of permissions and requirements refers to their epistemic versions. I also follow her in using \textit{required} and \textit{ought} interchangeably.}

\begin{quote}
    To make the tension between ZIP and [\ep{}] precise we should assume that `ought' and `permit' are duals, i.e., that one ought to $\phi$ at \textit{t} just in case one is not permitted to not-$\phi$ at \textit{t}. Let's also say that if a subject cannot both $\phi$ and $\psi$ at \textit{t}, then $\phi$-ing and $\psi$-ing are \textit{incompatible} for that subject at \textit{t}. And finally, if $\phi$-ing and $\psi$-ing are incompatible at \textit{t}, then we can say that $\phi$-ing at \textit{t} is a way of not-$\psi$-ing at \textit{t}, and $\psi$-ing at \textit{t} is a way of not-$\phi$-ing at \textit{t}. Altogether then, we can say that if you're required to $\phi$ at \textit{t}, and $\psi$-ing at \textit{t} is incompatible with $\phi$-ing at \textit{t}, then you're not permitted to $\psi$ at \textit{t}. (p. 14f)
\end{quote}
%
Friedman argues that \ep{} and ZIP are inconsistent because there are cases in which ZIP requires $\phi$-ing at \textit{t}, \ep{} permits $\psi$-ing at \textit{t}, but one cannot both $\phi$ and $\psi$ at \textit{t}---contradicting the above assumption. Here is such a case:

\begin{quote}
    For instance, imagine you're at a busy restaurant, and the dinner bill arrives. In order to figure out what you owe, you have to do some mental math. Maybe it’s possible to acquire some of the perceptual knowledge available to you in the restaurant while doing the math or to learn from your pre-existing evidence while adding and dividing, but at some point doing those calculations is going to require you to focus on that task. At that point, ZIP is going to place some restrictions on which evidence you are allowed to follow and what you are allowed to come to know. So again, if you ought to do your bill calculations now, then you’re not permitted to come to know all about (e.g.) the conversation at the next table now. (p. 16)
\end{quote}
%
The restaurant patron---call her Beth---has to figure out what she owes (\q{Q}). She may attend to her surroundings for some time, instead of focusing on \q{Q}, but there is going to be a point \textit{t} when Beth will be required to focus on \q{Q}. \ep{}, however, will still permit her to attend to the \q{Q}-irrelevant evidence available to her at \textit{t}. Friedman writes:

\begin{quote}
    In all of these cases we have ZIP declaring the making of some judgments impermissible, and, at the same time, [\ep{}] declaring the making of those same judgments permissible. We’ll be able to find tension like this in most any inquiry that requires our attention for any stretch of time (which I assume is most any inquiry). In those sorts of cases, there will typically be some traditionally epistemically impeccable judgments that it won’t be permissible to make. These judgments are then both permissible and impermissible. (ibid)
\end{quote}
%
The conclusion is that \ep{} and ZIP are inconsistent, and at least one of them will have to be rejected. It's important for what follows to lay out Friedman's argument for the inconsistency more precisely. I will focus on the argument for the conclusion that in Beth's restaurant case there are some judgments that are both permissible and impermissible. Given that Beth does not make sufficient progress on her inquiry \q{Q} for some time, and that her inquiry is temporally urgent, there comes a point \textit{t} at which:

\begin{enumerate}[label=(P\arabic*),ref=P\arabic*]
    \item\label{itm:prog} Beth has not made sufficient progress on \q{Q} at \textit{t}.
    \item\label{itm:evid} Beth has excellent evidence for some \q{Q}-irrelevant \textit{p} at \textit{t}.
    \item\label{itm:zip} If one has not made sufficient progress on a temporally urgent \q{Q} at \textit{t}, then one is required to \foc{}.\footnote{See p. 16: ``Maybe it’s possible to acquire some of the perceptual knowledge available to you [. . .], but at some point doing those calculations is going to require you to focus on that task.'', and p. 22: ``at some point in fairly typical inquiries inquirers will need to avoid distraction.'' See also fn. 21 in p. 15.} [ZIP]
    \item\label{itm:ep} If one has excellent evidence for \textit{p} at \textit{t}, then one is permitted to \jud{}. [\ep{}]
    \item\label{itm:cant} Beth cannot both \jud{} and \foc{}.
    \item\label{itm:opic} If one cannot both $\phi$ and $\psi$ at \textit{t} and one is required $\phi$ at \textit{t}, then one is not permitted to $\psi$ at \textit{t}.\footnote{This is the assumption stated in Friedman's p. 14 passage quoted above.}
\end{enumerate}
\begin{enumerate}[label=(C\arabic*),ref=C\arabic*]
    \item\label{itm:perm} Beth is permitted to \jud{}. [\ref*{itm:evid}, \ref*{itm:ep}]
    \item\label{itm:forb} Beth is not permitted to \jud{}. [\ref*{itm:prog}, \ref*{itm:zip}, \ref*{itm:cant}, \ref*{itm:opic}]
\end{enumerate}
%
\newcommand{\psub}{\ref*{itm:zip}, \ref*{itm:cant}, \ref*{itm:opic}}
%
Friedman then presents a dilemma: We'll have to reject \ep{} or ZIP. Both seem difficult to give up, but Friedman suggests that it is \ep{} that should go. As I shall argue, however, rejecting \ep{} is not a promising option, since premises \psub{} lead to a contradiction even if \ep{} is rejected.

\section{Inconsistent premises}
The following premises of Friedman's argument are inconsistent:

\begin{enumerate}
    \item[(\ref*{itm:zip})] If one has not made sufficient progress on a temporally urgent \q{Q} at \textit{t}, then one is required to \foc{}.
    \item[(\ref*{itm:cant})] Beth cannot both \jud{} and \foc{}.
    \item[(\ref*{itm:opic})] If one cannot both $\phi$ and $\psi$ at \textit{t} and one is required $\phi$ at \textit{t}, then one is not permitted to $\psi$ at \textit{t}.
\end{enumerate}
%
To see that \psub{} are inconsistent, note, first, that \ref*{itm:cant} entails:

\begin{enumerate}
    \item[(\ref*{itm:cant}')] If Beth judges \textit{p} at \textit{t}, she cannot \foc{}.
\end{enumerate}
%
\ref*{itm:cant}' follows from \ref*{itm:cant} given the assumption that one cannot undo one's actions---once one $\phi$'s at \textit{t}, it is an immutable fact that one $\phi$-ed at \textit{t}. If Beth judges \textit{p} at \textit{t}, either she can also \foc{} or she can't. If she can, then it will be possible for her to have both judged \textit{p} at \textit{t} and focused on \q{Q} at \textit{t}---in contradiction to \ref*{itm:cant}. Thus, if Beth judges \textit{p} at \textit{t}, she cannot \foc{}.

\ref*{itm:opic}, on the other hand, entails an \textit{ought implies can} thesis:

\begin{enumerate}
    \item[(OC)] If one is required to $\phi$ at \textit{t}, one can $\phi$ at \textit{t}.
\end{enumerate}
%
To see that OC follows from \ref*{itm:opic}, consider the case where $\phi$ = $\psi$. Suppose for reductio that one is required to $\phi$ but cannot $\phi$. Since one cannot $\phi$ and is required to $\phi$, \ref*{itm:opic} entails that one is not permitted to $\phi$, in contradiction to the assumption that one is required to $\phi$.

Given the assumption that judging \textit{p} at \textit{t} is consistent with not having made sufficient progress on \q{Q} at \textit{t}, we have a contradiction. If Beth judges \textit{p} at \textit{t}, and she has not made sufficient progress on \q{Q} at \textit{t}, then:

\begin{enumerate}[label=(C\arabic*'),ref=C\arabic*]
    \item Beth is required to \foc{}. [\ref*{itm:zip}]
    \item Beth cannot \foc{}. [\ref*{itm:cant}']
    \item Beth is required to \foc{} only if she can \foc{}. [OC]
\end{enumerate}
%
C1'-C3' are clearly inconsistent. Since they are entailed by \psub{}, together with the assumption that it is possible to \jud{} when one has not made sufficient progress on \q{Q} at \textit{t}, one of those will have to be rejected. There is no reason to reject the latter assumption---indeed, without it, the claim that ZIP requires that one not \jud{} does not make much sense---so one of \psub{} must go.

Since rejecting \ref*{itm:cant}---the claim that one cannot both focus on an inquiry and form an unrelated belief at the same time---does not seem very plausible, it is down to \ref*{itm:zip} and \ref*{itm:opic}. In fact, there is another reason to think that \ref*{itm:zip} and \ref*{itm:opic} are inconsistent. As Friedman grants, one can have more than one inquiry going on at a time.\footnote{\textcite[10]{friedman_epistemic_nodate}.} For instance, I may need to figure out both whether I can afford a certain apartment (\q{Q\textsubscript{1}}), and whether I need to restock my vitamins (\q{Q\textsubscript{2}}), and it is possible that both are urgent---that I need to figure out both by tonight. But given \ref*{itm:opic}, there can be no \textit{t} such that I am required to focus on \q{Q\textsubscript{1}} at \textit{t} and required to focus on \q{Q\textsubscript{2}} at \textit{t}, even if I make no progress at all on either inquiry. For \ref*{itm:opic} entails a joint satisfiability thesis:

\begin{enumerate}
    \item[(\ref*{itm:opic}')] If one is required to $\phi$ at \textit{t} and required to $\psi$ at \textit{t}, one can both $\phi$ and $\psi$ at \textit{t}.\footnote{To see that \ref*{itm:opic} entails \ref*{itm:opic}', note that \ref*{itm:opic} is equivalent to the following: If one is required to $\phi$ at \textit{t} and permitted to $\psi$ at \textit{t}, one can both $\phi$ and $\psi$ at \textit{t}. Given that \textit{required to $\phi$} entails \textit{permitted to $\phi$}, \ref*{itm:opic}' follows.}
\end{enumerate}
%
Since there is no \textit{t} such that I can both focus on \q{Q\textsubscript{1}} and focus on \q{Q\textsubscript{2}} at \textit{t}, either \ref*{itm:opic}', and thus \ref*{itm:opic}, is false, or having made no progress on a temporally urgent inquiry \q{Q} at \textit{t} is not sufficient for ZIP to require that one \foc{}, and thus \ref*{itm:zip} is false.

Friedman's case for the inconsistency between \ep{} and ZIP relies on an argument that \ep{} and ZIP together with some other assumptions lead to a contradiction. Like any \textit{reductio}, the argument only shows that at least one of its premises must be rejected; it does not tell us which one it is. Since a contradiction remains even without \ep{}, it is surely not \ep{} that Friedman's argument warrants rejecting.

\section{The structure of evidential norms}

Suppose that \ep{} is nevertheless rejected, and that Friedman's argument for the inconsistency is sound. Must we conclude that all non-zetetic epistemic norms should be rejected as well, as Friedman suggests? In this section, I shall argue that there are nearby epistemic norms that are both compatible with ZIP and more plausible than the the epistemic norms which Friedman argues are inconsistent with ZIP.

Assuming a qualitative view of belief, there are four possible doxastic states given a proposition \textit{p}:

\begin{enumerate}
    \item Believing \textit{p}
    \item Disbelieving \textit{p}
    \item Suspending judgment on \textit{p}
    \item Having no doxastic attitude toward \textit{p}\footnote{I am assuming with \textcite{friedman_why_2017} that 3 and 4 are distinct states.}
\end{enumerate}
%
One way to think about evidential norms of the sort of \ep{} is that they tell us which of 1-4 we are required, or permitted, to have in a given situation. Another is that their advice is limited to 1-3: they never require, or issue a permission, for one to make the transition from 4 to 1-3. But given that one does make that transition, they tell us which of 1-3 are permitted or required. The following correspond to the former and latter kind of norms:

\begin{description}
    \item[\eo] If one has excellent evidence for \textit{p} at \textit{t}, then one ought to \jud{}.
    \item[\eoc] If one has excellent evidence for \textit{p} at \textit{t}, then one ought to \judif{}.\footnote{In fact, \textcite[679]{feldman_ethics_2000}, who Friedman cites as one of the advocates of \eo{}, accepts something like \eoc{} instead.}
\end{description}
%
For instance, if I have excellent evidence that my neighbours are home, a norm of the former kind will require me to attend to that evidence and to judge that my neighbours are home. A norm of the latter kind will not; it will only require me to judge they are home given that I take some doxastic attitude toward that proposition. The former norm, a stronger version of \ep{} is, according to Friedman, inconsistent with ZIP. The latter, I shall argue, is both more plausible than \eo{} and consistent with ZIP.

Norms of the former kind are much less plausible for suspensions of judgment. Consider:

\begin{description}
    \item[\so] If one's evidence supports neither \textit{p} nor $\neg $\textit{p} at \textit{t}, then one ought to suspend judgment on \textit{p} at \textit{t}.
    \item[\soc] If one's evidence supports neither \textit{p} nor $\neg $\textit{p} at \textit{t}, then one ought to suspend judgment on \textit{p}, if one takes any doxastic attitude toward \textit{p} at \textit{t}.
\end{description}
%
There are infinitely many propositions that I have never considered, that I have no doxastic attitude toward, and for or against which I have no evidence at this moment. \so{} requires that I take a doxastic attitude---suspension of judgment---toward all of these, and all at once. This is implausible, especially if we are assuming with Friedman the following thesis (previously \ref{itm:opic}'):

%
\newcommand{\js}{Joint Satisfiability}
\begin{description}
    \item[\js{}] If one ought to $\phi$ at \textit{t} and ought to $\psi$ at \textit{t}, one can both $\phi$ and $\psi$ at \textit{t}.
\end{description}
%
\js{} entails that \so{} is false, as suspending judgment about some proposition will require that one at least consider it, and one cannot do so for two propositions simultaneously. \js{} is also inconsistent with \eo{}, given that one can have evidence for more than one proposition at the same time. If at \textit{t} I have both evidence that my neighbours are home and evidence that I have thirty-one browser tabs open, then \eo{} requires that I follow both at \textit{t}. Since I can't do both at the same time, \eo{} is incosnistent with \js{}.

\eoc{} and \soc{} are not susceptible to this problem. \soc{} doesn't require that one suspend judgment on every one of the infinitely many propositions that one has never considered, and for or against which there is no evidence. Rather, it requires that one suspend judgment on those propositions one actually considers and toward which one takes some doxastic attitude. One violates \soc{} by believing or disbelieving propositions for or against which one lacks evidence. One doesn't violate it by having no doxastic attitude toward such propositions. Similarly, \eoc{} requires that one form a belief about propositions one actually considers. If I never consider whether my neighbours are home, and take no doxastic attitude toward that proposition, then I do not violate \eoc{} even if I have excellent evidence that they are home. But once I do attend to that proposition, \eoc{} requires that I judge in accordance with the evidence.

To be clear, norms such as \eoc{} and \soc{} do not imply that it is epistemically permissible to not transition from not having to having a doxastic attitude. Rather, their advice is limited to cases in which a doxastic attitude is taken, and they are compatible with the existence of other norms that do have something to say about the permissibility of such transitions.

Importantly, \eoc{} is compatible with ZIP. Consider Beth's restaurant bill inquiry \q{Q}. Suppose that ZIP requires that she focus on \q{Q} at \textit{t}, and at the same time, she has excellent evidence that there are five people sitting at the table to her right (\textit{p}). While \eo{} required that Beth judge \textit{p} at \textit{t}, and thus conflicted with ZIP and \js{}, \eoc{} does not. The requirement to focus on \q{Q} at \textit{t} is compatible with \eoc{}'s requirement to judge \textit{p} if Beth takes some doxastic attitude toward \textit{p}. Beth will have satisfied ZIP and not violated \eoc{} by focusing on \q{Q} and taking no doxastic attitude toward \textit{p}.

If \eoc{} is compatible with ZIP, then so is a weaker permission version of it:

\begin{description}
    \item[\epc] If one has excellent evidence for \textit{p} at \textit{t}, then one is permitted to \judif{}.
\end{description}
%
\epc{} is weaker than \ep{}. While \ep{} permitted to transition from having no doxastic attitude to having some doxastic attitude, given one's evidence, \epc{} says nothing about the permissibility of such transitions. \epc{} only says that judging \textit{p} is permitted given that one makes the transition. In Beth's case, \epc{} doesn't say that Beth is permitted to attend to the available evidence about nearby tables, so it does not clash with a requirement to focus on \q{Q}.

The conditional evidential norms \eoc{}, \soc{}, and \epc{} are both compatible with ZIP and independently more plausible than their unconditional counterparts. Thus we need not accept Friedman's conclusion that all non-zetetic epistemic norms ought to be rejected, even if we accept her case for the incompatibility of ZIP and unconditional evidential norms.

One might worry that there is a significant cost to rejecting a norm like \ep{}, and giving up the idea that we are always permitted to judge in accordance with the evidence. As Friedman puts it,

\begin{quote}
It [means] denying that there are blanket epistemic permissions to follow our evidence and blanket epistemic permissions to come to know. This leaves it open that there are cases in which believing in accordance with the evidence and/or coming to know are not epistemically permissible.

Those sorts of blanket permissions are central to normative epistemology as we know it though, and so rejecting them should force a fairly significant rethink of our current understanding of epistemic normativity. (p. 29)
\end{quote}
%
I shall argue, however, that this worry is overstated.

\section{What cost?}

What is the cost of giving up the idea that there are blanket permissions to judge in accordance with the evidence and to come to know? Friedman writes that this ``leaves it open that there are cases in which believing in accordance with the evidence and/or coming to know are not epistemically permissible'' (ibid). To flesh out this worry a bit more, consider Beth, who is required to \foc{}. Suppose that instead she comes to know that there are five people sitting at the table to her right (\textit{p}) at \textit{t}. Since, according to ZIP, Beth is not epistemically permitted to come to know \textit{p} at \textit{t}, her resulting state of knowledge is thus epistemically impermissible. But this is an unsettling thought. How can one's state of knowledge be epistemically impermissible? This conflicts, at the very least, with the idea that knowledge is always epistemically good. As Friedman writes: ``It’s hard to know quite how to think about epistemically problematic knowledge or knowledge acquisition on our current understanding of normative epistemology'' (ibid).

I shall argue, however, that Beth's resulting knowledge that \textit{p} is permissible, even if her coming to know \textit{p} was not permissible. First, one cannot both accept ZIP and plausibly claim that Beth's knowledge that \textit{p} is impermissible. Suppose that at \textit{t} Beth knows that \textit{p} as a result of impermissibly coming to know \textit{p}, which she did instead of focusing on \q{Q} as she was required by ZIP. If Beth is not permitted to know \textit{p} at \textit{t}, then, since requirements and permissions are duals, Beth is required not to know \textit{p} at \textit{t}. Either it is possible for Beth not to know \textit{p} at \textit{t}, or it isn't. If it isn't, then by the \textit{ought implies can} which Friedman's argument assumes (OC above), Beth is not required not to know \textit{p} at \textit{t}, and thus is permitted to know \textit{p} at \textit{t}. If, on the other hand, it is possible for her not to know \textit{p} at \textit{t}, then she is required to do that and lose her knowledge in some way. But that requires not focusing on \q{Q}, which would violate ZIP. Since ZIP requires Beth to \foc{}, and Beth cannot both \foc{} and lose her knowledge that \textit{p} at \textit{t}, Friedman's own \ref*{itm:opic} entails that Beth is \textit{not} permitted to lose her knowledge that \textit{p} at \textit{t}. So she is required not to lose it, and thus she is permitted to know \textit{p} at \textit{t}. Either way, then, Beth's resulting knowledge is permissible, even if her coming to know was not.

Second, the inference that Beth's knowledge that \textit{p} is not permissible because her coming to know \textit{p} was not permissible seems to be based on the following:

\newcommand{\pp}{Permission Transfer}
\newcommand{\ppk}{\pp{\textsubscript{K}}}
\newcommand{\ppb}{\pp{\textsubscript{B}}}
\begin{description}
    \item[\ppk] If one knows \textit{p} as a result of coming to know \textit{p}, then: one is permitted to know \textit{p} only if one was permitted to come to know \textit{p}.
    \item[\ppb] If one believes \textit{p} as a result of judging \textit{p}, then: one is permitted to believe \textit{p} only if one was permitted to judge \textit{p}.\footnote{In her \parencite*[p.~689f]{friedman_teleological_2019}, Friedman argues for a thesis equivalent to \ppb{}, writing that ``if some judgment is impermissible then the resulting belief state is as well.''}
\end{description}
%
\ppk{} and \ppb{} are instances of a more general thesis:

\begin{description}
    \item[\pp] If one is in state S as a result of $\phi$-ing, then: one is permitted to be in S only if one was permitted to $\phi$.
\end{description}
%
But \pp{} is false. If a norm deems it impermissible to transition from state S\textsubscript{1} to state S\textsubscript{2}, it does not have to deem it impermissible to remain in state S\textsubscript{2} once there. Suppose I am not permitted to exit my house (A), because of a lockdown that forbids being outside. I am thus also not permitted to enter the house across the street (B), since that would require exiting my house. But if I do move from A to B, violating the lockdown, it is not a violation of the lockdown to remain in B. On the contrary, exiting B would be a further violation of the lockdown.

We should distinguish the permissibility of entering a state and the permissibility of remaining in a state. As the above example shows, remaining in a state S could be permitted even if entering S was not. This is exactly what happens in Beth's case: she was not permitted to come to believe \textit{p}, but she is permitted to retain her belief that \textit{p}. The permissibility of acquired belief or knowledge states is entirely unaffected by rejecting \ep{} and \eo{} in favor of \epc{} and \eoc{}. Friedman's argument, then, does not have the significant consequences it is claimed to have.

\printbibliography

\end{document}
